\documentclass[a4paper]{moderncv}
\moderncvtheme[blue]{classic}

\usepackage[scale=0.86]{geometry}
\usepackage{soul}
\usepackage{xcolor}
\graphicspath{{Files/}{../}{./}}
\usepackage{fontawesome5}\newcommand{\topic}[1]{\textbf{\textsc{#1}}}
\newcommand{\eubadge}{\raisebox{-0.25ex}{\colorbox[HTML]{003399}{\textcolor[HTML]{FFCC00}{\scriptsize *****}}}}

\firstname{Iraklis}
\familyname{Symeonidis}
\title{Senior Cybersecurity Architect \textbackslash{} PhD, MSc, ISO/IEC 27001 Certified}
\email{iraklis.symeonidis@icloud.com}
\homepage{https://isymeonidis.github.io}
\social[linkedin]{linkedin.com/in/iraklis-symeonidis}
\address{Stockholm, Sweden}{}{}
\mobile{+46 73-686 22 07}
\photo[40pt]{/Users/iraklissy/Developer/2.Dev-local/Webpage-isymeoni/isymeonidis.github.io/Files/Isymeonidis\_profile\_image.jpeg}

\begin{document}
\maketitle

\section{Professional Summary}
Senior Research Scientist at RISE and cybersecurity professional with over a decade of experience securing critical infrastructure and leading teams within complex, multi-stakeholder environments. Expert in coordinating technical and regulatory teams to bridge the gap between engineering execution and strategic governance, ensuring robust alignment with ISO/IEC 27001, GDPR, and emerging EU regulations. Proven ability to translate regulatory frameworks into practical, engineering-driven solutions for cross-domain environments. Recognized for orchestrating secure architecture design, cyber resilience, and governance by aligning cross-functional stakeholders to drive informed, risk-based decision-making. Collaborative team player and communicator, dedicated to fostering a security-first culture and empowering teams to navigate the evolving cybersecurity landscape.

\section{Technical \& Leadership Skills}

\noindent
\begin{minipage}[t]{0.49\textwidth}
    \textbf{Technical Skills}\par\medskip
        Cyber Resilience and Secure Architectures \newline
        Security-by-Design and DevSecOps \newline
        AI/ML Security and OT/IT Industrial Security \newline
        ISO/IEC 27001 Lead Auditor Certified \newline
        Regulatory Awareness (GDPR, AI Act, CRA)
    \end{minipage}\hfill
    \begin{minipage}[t]{0.49\textwidth}
    \textbf{Leadership Skills}\par\medskip
        Cybersecurity Architectural Guidance \newline
        Cross-Functional Coordination \newline
        Technical Guidance and Mentoring \newline
        Security Culture and Team Enablement \newline
        Governance and Strategic Decision-Making
\end{minipage}

\section{Professional Experience}

\cventry{2022 -- Present}{Cybersecurity Architect and Senior Researcher}{RISE Research Institutes of Sweden}{}{}{
- Lead secure architecture design for critical infrastructure (Automotive, Telecom, Manufacturing) by embedding security-by-design and DevSecOps principles. \newline
- Direct cybersecurity strategy and governance, ensuring alignment with ISO/IEC 27001, GDPR, the AI Act, and emerging EU regulations. \newline
- Secure and lead industrial collaborations (Horizon Europe, Vinnova), translating advanced research into applied industrial outcomes. \newline
- Mentor engineering teams on threat modeling and risk assessment to foster a resilient security culture.
}

\cventry{2020 -- 2022}{Scientist, Secure Communications and Cybersecurity}{KTH Royal Institute of Technology}{}{}{
- Designed secure architectures for connected mobility systems with emphasis on resilience, integrity, and safety-critical communication. \newline
- Modeled cybersecurity requirements and system behavior for distributed vehicular and edge-based architectures. \newline
- Mentored researchers and students in secure networking, applied cryptography, and security engineering.
}

\cventry{2018 -- 2020}{Scientist, Secure Communications and Architectures}{University of Luxembourg}{}{}{
- Evaluated security of enterprise communication platforms with focus on end-to-end encryption and data protection. \newline
- Supported architecture reviews for secure identity, authentication, and data exchange systems.
}

\cventry{2012 -- 2018}{Scientist, Secure Systems Architectures}{KU Leuven (COSIC)}{}{}{
- Led research on large-scale ecosystem security risks, including analysis of third-party application vulnerabilities (Cambridge Analytica case). \newline
- Designed secure data-sharing, key management, and dynamic access management architectures applied in automotive ecosystems.
}

\section{Selected Projects \& Research}

\cventry{2023 -- Present}{\topic{Automotive, Manufacturing, Supply Chain and Cybersecurity}}{}{}{}{
    I. \textbf{TWINLOOP (Horizon Europe):} Secure Systems Architecture Lead for a Zero-Trust Digital Twin framework for software-defined electric vehicles. Designed reference security architecture (vehicle-edge-cloud), led threat modeling, and defined controls for cross-OEM interoperability. Security principles and modeling for distributed digital twin operations in automotive and industrial settings. \newline
II. \textbf{Manufacturing OT/IT Cybersecurity:} Contributed to risk assessments, OT/IT threat modeling, and secure automation practices in industrial environments. \newline
III. \textbf{Datadelning i en digitaliserad köttkedja (FORMAS):} Research on cybersecurity, privacy, and secure data-sharing mechanisms for industrial digitalization and supply-chain transformation.
}

\cventry{2023 -- Present}{\topic{Next-Generation Networks and Cybersecurity}}{}{}{}{
    I. \textbf{CitCom.ai (Digital Europe and Vinnova):} Contribution on cybersecurity, resilience, and connectivity for Data Space connector services. Designed and validated cybersecurity components including identity-based access control, security for data at transfer and at rest for edge/IoT platforms. \newline
II. \textbf{BEiNG-WISE (COST Action):} Working Group Leader coordinating research on cybersecurity in next-generation wireless systems (5G/6G), including workshops, training, and cross-disciplinary collaboration.
}

\cventry{2025 -- Present}{\topic{Sovereignity, Governance and Cybersecurity}}{}{}{}{
    I. \textbf{NexusForum (Horizon Europe):} Working Group Co-leader of the cybersecurity working group, contributing to the European Research and Innovation Roadmap on secure cognitive computing continuum ecosystems. \newline
II. \textbf{AI/ML Security Threats:} Research on cyber risks emerging from AI-driven and autonomous systems, focusing on robustness, model integrity, and alignment with EU regulatory frameworks (AI Act). Governance and risk modeling contributions across EU-funded projects assessing AI system reliability, secure deployment pipelines, and compliance for safety-critical functions.
}

\section{Degree Qualifications}
\cventry{2018}{PhD in Secure and Privacy-Preserving Systems Architectures (automotive focus)}{KU Leuven}{Belgium}{}{}
\cventry{2013}{MSc in Digital Systems Security}{University of Piraeus}{Greece}{}{}
\cventry{2004}{Diploma in Information and Communication Systems Engineering}{University of the Aegean}{Greece}{}{}

\section{Certifications \& Training}
\cventry{2013}{ISO/IEC 27001 Lead Auditor}{TÜV NORD}{}{}{}
\cventry{2014}{Advanced Cyber Security Course}{ENCS, The Hague}{}{}{}
\cventry{2014}{Secure Application Development}{SecAppDev, Belgium}{}{}{}
\cventry{2007}{\href{http://www.kphgraz.at/}{``Modern pedagogical and teaching methods in education''}}{KPH Graz}{Austria}{}{}

\section{Languages}
English (Fluent), Greek (Native), Swedish (A2)

\section{References}
References upon request

\end{document}
